\documentclass[a4paper, 12pt]{article}
\usepackage[left=2cm,right=2cm,top=2cm,bottom=2cm]{geometry}
\setlength{\parindent}{1cm}
\usepackage{graphicx}
\begin{document}

\title{MM2090 Assignment-4}
\author{Alpha P Jose ME20B021}
\date{June 2021}
\maketitle

\section{Alpha P Jose ME20B021}

Electric field equation for a point charge :  

\begin{equation}
 {\LARGE{\textbf{$E =\frac{kQ}{{r^2}}$}}}
 \label{eqn1:equation1}
\end{equation}


\subsection{Analysis}
The following paragraph contains a brief explanation of the variables and  importance of the equation :
\begin{itemize}

    {\normalsize {The above given equation \ref{eqn1:equation1}  has the following terms \textbf{E} ,\textbf{k} , \textbf{Q} and \textbf{r}.}}

{\normalsize { Here,}}\\
{\normalsize {\textbf{E} represents electric field strength }}\\
{\normalsize {\textbf{k} \  represents the Coulombs Constant}}\\
{\normalsize {\textbf{Q} \  represents the point charge producing the field}}\\
{\normalsize{\textbf{r} \ represents the distance from point charge}}
\end{itemize}


Electric field equation for a point charge an experimental ~\ref{eqn1:equation1} of physics that allows us to calculate intensity of the electric field produced at a known distance 'r' by a point charge . Orginally derived from the Coloumbs force equation where $F=qE$ ,this law was first discovered in 1762 by English physicist Lorentz .This is the electric field at point due to the point charge Q which is equivalent  to the Coulomb force per unit charge that a point charge would experience at a position .This equation derived for electric field of a point charge as it made it possible to discuss the Electric field produced by a point charge in a meaningful way.

\begin{figure}[h]
	{\begin{center}
		\includegraphics[scale=0.4]{ME20b021.jpg}
	\end{center}}
	\caption{Electric field for a point charge~\cite{picture}}
	\label{f1:image1}
\end{figure}

Webpage Links ~\cite{website}



%\bibliography{bibliography.bib}
%\bibliographystyle{plain}

\end{document}
